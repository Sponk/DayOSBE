\chapter{Operating System Interfaces}
\section{Registering and Querying a Service}
\subsection{What is a Service?}
Since DayOS is using a micro kernel all operating system related tasks run in external user mode processes without any access to internal kernel structures.
This allows for a great deal of security and some other goodies related to it but brings a few limitations and problems with it.
One of the problems is that all OS related subsystems run in processes that are not fixed, that means
variation of attributes like the Process ID (PID) or even variations in the implementation.
To make it possible for applications to talk to any non-kernel component the program has
to know the other processes PID to send and receive messages. This is the main use-case for the Service Manager (SM).
It allows registering a service to a name with a PID and allows querying the PID of the process
that was registered for this service. \\ \\
The SM has to run in the kernel to be able to check if a process is still running or if it crashed
to allow restarting the service. As soon as there are interfaces to the kernel functionality required this daemon should run
in user mode.
\subsection{Registering a Service}
Registering a service is specified as a message to to the SM with the signal \texttt{SERVICE\_REGISTER = 1}.
Since the SM is running as a kernel process it has a fixed PID = 0. The message body is undefined and does not
produce any side effects. If the request was successful the SM sends a message with \texttt{MESSAGE\_OK} back to you.
If it failed a message with \texttt{MESSAGE\_FAIL} is sent.

\subsection{Querying a Service}
Querying a serice is specified as a message to the SM with the signal \texttt{SERVICE\_REQUEST = 2}.
The SM in turn will then send a message with the signal being the PID of the requested service or \texttt{MESSAGE\_FAIL}
if no such service was found.

\section{The Virtual File System}
The Virtual File System (VFS) in DayOS is a fundamental part of the system.
It takes responsibility of managing (device-)files, directories, links and mount points
for all applications. The VFS runs in user mode and has in no way access to internal
kernel structures. After the process starts it immediately has to register itself as
the "vfs" service. During run time the VFS process has to respond to certain signals
in a certain manner. \\ \\
The VFS transfers data using the following structures:

\verbatiminput{../libs/libdayos/include/vfs.h}

The VFS has to support the following signals: \\
\begin{itemize}
\label{create_device}
\item VFS\_SIGNAL\_CREATE\_DEVICE = 0\\
  Creates a new device file and adds it to the VFS internal list of files.
  A device file can be one one of the following types:

  \begin{enumerate}
    \item VFS\_CHARACTER\_DEVICE = 0
	\item VFS\_BLOCK\_DEVICE = 1
	\item VFS\_MOUNTPOINT = 2
	\item VFS\_DIRECTORY = 3
  \end{enumerate}

  This signal has a vfs\_request structure in its message body with all needed data to actually
  create the device file internally. How this is handled internally is not specified. \\ \\
  When the file could be created, send a message with signal "SIGNAL\_OK" to the requesting process and
  the signal "SIGNAL\_FAIL" on failure.

\item VFS\_SIGNAL\_OPEN = 1 \\
  This signal gets a vfs\_request structure in its message body.
  On success it results in a message with signal "SIGNAL\_OK" and a vfs\_file structure in the body to be sent
  to the requesting process. On failure a message with "SIGNAL\_FAIL" is being sent.

\item VFS\_SIGNAL\_MOUNT\_DEVICE = 2 \\
  Used to mount devices represented as device files to any location in the VFS. This action needs
  superuser access (The calling process has to have the User ID (UID) 0).
  If a prerequisites are met the VFS sends a mount request to the FS driver that was requested by
  the caller. %% TODO: Need specification for that!

\end{itemize}

\section{The Driver Interface}
\subsection{General Interface}
In DayOS every device is represented as a file in the VFS tree. Every driver has to deliver
the abstract interface dictated by this design to be able to handle all requests that may
be sent. When a driver starts and finishes his initialisation routine a device file has to be
created. This procedure is described in the section \ref{create_device} about the VFS. \\ \\
In the most general terms a driver has to respond to at least three different signals:
DEVICE\_READ, DEVICE\_WRITE and DEVICE\_INFO. Those however have to be handled differently
for different types of device.

\subsection{Data Interchange Structures}
The following structures are used to transfer data between drivers and applications.
\begin{verbatim}
#define REQUEST_MAGIC 0xDEADBEEF

typedef enum
{
	FS_SIGNAL_MKDIR = 0,
	FS_SIGNAL_OPEN = 1,
	FS_SIGNAL_CLOSE = 2,
	FS_SIGNAL_READ = 3,
	FS_SIGNAL_WRITE = 4,
	FS_SIGNAL_MOUNT = 5,
	FS_SIGNAL_STAT = 6
} FS_SIGNALS;

typedef enum
{
	DEVICE_READ = 0x1,
	DEVICE_WRITE = 0x2
} DEVICE_ACTIONS;

struct device_request
{
    uint32_t magic;
    uint32_t size;
    uint8_t character;
};

struct fs_request
{
    uint32_t magic;
    FS_SIGNALS signal;    
};

\end{verbatim}

\subsection{Character Devices}
Character devices are devices that handle data one character at a time. This is the reason why
a character device in DayOS answers to DEVICE\_READ requests with one character per message.
Every driver should have an internal buffer so no data is lost between requests. If the
buffer is empty the driver queues the pending request and handles it as soon as possible.
Writing to a character device also works on a per character basis but it is possible to
send full messages that are going to be handled at once without the need to resend a message
for every character.

%% TODO: Should char devices be able to send their whole buffer at once?

\subsection{Block Devices}
Block devices can only read and write data in fixed block sizes instead of per character.
That means that requesting data can exceed the size of one message very easily and has
to split up to multiple packets. The protocol to transfer data (in both directions)
is defined like following: \\
\begin{enumerate}
\item Initial request with the signal DEVICE\_READ and a device\_request in the body.
\item Wait for answer from the opposite side. The signal is SIGNAL\_OK or SIGNAL\_FAIL.
\item Now the opposite side will send a load of messages with the signal being the size sent.
\item If the message has a size of 0, quit the loop. If the message has a size > 0, continue at step 3.
\end{enumerate}

\subsection{File System Drivers}
File System Drivers (FSDs) are among the few drivers that depend heavily on other drivers.
They also do not create device files, but only provide mountpoints. When a device is mounted
the requesting process sends a request to the VFS. The VFS looks for the device file and
sends the file system driver that was requested a message. The FSD will then open the device file and
create a mount point in the VFS if the File System (FS) on the device is the right type the FSD
can read/write. The VFS will then create the internal mountpoint so future accesses to files
will recognize the mounted device. The procedure works like as follows: \\
\begin{enumerate}
\item User application sends a vfs\_request to the VFS service with the \\
VFS\_SIGNAL\_MOUNT\_DEVICE signal. The first message has the source device in the path. 
A second message with the target directory in the path attribute
and the requested filesystem driver PID in the argument attribute. All FSs register with the service manager.

\item VFS service looks if the requested device exists. If not, the application receives a SIGNAL\_FAIL message.
Otherwise the VFS service requests a mount at the given FSD PID. If a message returns with SIGNAL\_OK, the mount point
is being created. Otherwise the application receives a SIGNAL\_FAIL message.

\item The FSD receives a request from the VFS to mount a certain device with a fs\_request in the message body.
\end{enumerate}

\subsection{Graphical Devices}

\subsection{Transmission of Environment Variables and Arguments}
When an application is started it gets an environment and command line arguments passed to it.
This chapter is used to specify how this is done on the side of the runtime library. This features does
not need the kernel in principle given that parent process and child process use the same
functions to transmit the needed data. \\ \\
The protocol is very simple:
\begin{enumerate}
 \item A message with signal = SIGNAL\_OK and size = argc to begin transmission.
 \item A message with signal = SIGNAL\_OK and message = argv[iteration]
 \item If still any argument is left, go to step 1, else continue.
 \item A message with signal SIGNAL\_OK and message = variable name
 \item A message with signal SIGNAL\_OK and message = variable value
 \item If still any environment variable is left, go to step 4, else continue.
 \item A message with signal SIGNAL\_OK and size = 0 to end transmission.
\end{enumerate}
Keep in mind that the size field of the message always needs to be greater than 0 or else the transmission will terminate.
It should be enough to set it to argc since at least one argument needs to be sent anyways.

