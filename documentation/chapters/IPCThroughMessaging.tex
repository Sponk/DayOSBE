\chapter{Inter Process Communication Through Messaging}
\section{The Abstract Idea}
When two or more processes need to share data, inform each others about changes or simply want to interchange some form
of data, the easiest way in DayOS is message passing. The idea is that each process possesses some sort of
"in-box" in which received messages are saved until the process explicitly looks for them.
When sending a message the sending process has to fill up a structure containing an integer for some sort of
signal and a character array of a predefined size ($MESSAGE\_STRING\_SIZE = 512$ by default on IA-32) which can be used
for all sorts of data, be it ASCII strings, UTF-8 strings or binary data. In the case that not the whole message
string is used there is a size attribute which specifies the size to be transferred to the receiving process. \\ \\
It is very easy to abstract interrupts and callbacks in this fashion, it even allows for simple forms of Remote Procedure Calls (RPC).
It is also a viable method of transferring data from some sort of disk driver to the file system driver or
file data from the file system to an user process.

\section{Implementation In The DayOS Kernel}
The message IPC system is one of the few pieces of software that run inside the kernel in a privileged position so it can copy over
data from one process to another without causing exceptions. The implementation consists of a doubly linked list for each process which
contains a FIFO queue of messages received. When a process calls the syscall to request the next message of a given sender process,
the FIFO is searched for the first message that was sent by the specified process or an error when no message could be found.
The kernel function to receive a message is called \\ \\
\texttt{uint32\_t kreceive\_message(process\_t* receiver, message\_t* message, uint32\_t who)}. \\
The arguments are pretty much self explanatory, the receiving process (in the case of a system call the currently running process)
is the first argument, the second one is a pointer to a valid message\_t object from the currently running process (can be on the stack or
on the heap) and who is the Process Identification (PID) from the sending process. The receiver is checked to be a valid process and
the message address is being checked to not point to a kernel mode or un-mapped memory address to prevent overwriting sensitive data.